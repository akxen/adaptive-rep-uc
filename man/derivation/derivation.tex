\documentclass{article}

\setlength{\parindent}{0pt}
\setlength{\parskip}{8pt}

\usepackage{notation}
\usepackage{geometry,amsmath,amssymb,bm,mathtools,graphicx,pdfpages,tikz,float,blkarray,longtable,titlesec,cases,bm,titlesec}
\graphicspath{{figures/}}
\setcounter{secnumdepth}{4}

\titleformat{\paragraph}
{\normalfont\normalsize\bfseries}{\theparagraph}{1em}{}
\titlespacing*{\paragraph}
{0pt}{3.25ex plus 1ex minus .2ex}{1.5ex plus .2ex}

\geometry{
	a4paper,
	total={170mm,257mm},
	left=15mm,
	top=30mm,
	right=15mm,
	bottom=30mm
}

\title{MPC updating for Refunded Emissions Payment Schemes\\Unit Commitment Formulation}

% Operators
\DeclareMathOperator*{\minimise}{minimise}

\begin{document}
	\maketitle
	

\section{Notation}
\renewcommand*{\arraystretch}{1.3}
\begin{longtable}{ p{.09\textwidth}  p{.75\textwidth}}
	\textbf{Symbol} & \textbf{Description}\\
	\hline\hline
	\multicolumn{2}{l}{\textbf{Indices}}\\
	$\iGenerator$ & Generator\\ 
	$\iScenario$ & Scenario\\
	$\iInterval$ & Interval\\
	$\iIntervalAlias$ & Interval alias\\
	$\iIntervalTerminal$ & Final dispatch interval\\
	$\iCalibrationInterval$ & Calibration interval\\
	$\iCalibrationIntervalAlias$ & Calibration interval alias\\
	$\iZone$ & NEM zone\\
	$\iRegion$ & NEM region\\
	$\iLink$ & Link connecting adjacent NEM zones\\
	& \\
	\multicolumn{2}{l}{\textbf{Sets}}\\
	$\sGenerators$ & All generators\\
	$\sGeneratorsThermal$ & Thermal generators\\
	$\sGeneratorsHydro$ & Hydro generators\\
	$\sStorage$ & Storage units\\
	$\sGeneratorsEligible$ & Generators covered by emissions abatement scheme\\
	$\sIntervals$ & Time intervals\\
	$\sZones$ & NEM zones\\
	$\sRegions$ & NEM regions\\
	$\sGeneratorsThermalQuickStart$ & Quick-start thermal generators\\
	$\sGeneratorsThermalSlowStart$ & Slow-start generators\\
	$\sLinks$ & Links connecting NEM zones\\
	$\sCalibrationIntervals$ & Calibration intervals\\
	& \\
	\multicolumn{2}{l}{\textbf{Variables}}\\
	$\vBaseline$ & Emissions intensity baseline applying to calibration interval $\iCalibrationInterval$ [tCO$_{2}$/MWh]\\
	$\vPermitPrice$ & Permit price applying to calibration interval  $\iCalibrationInterval$ [\$/tCO$_{2}$]\\
	$\vOnIndicator$ & Generator on indicator [--]\\
	$\vStartupIndicator$ & Generator startup indicator [--]\\
	$\vShutdownIndicator$ & Generator shutdown indicator [--]\\
	$\vPower$ & Power output above minimum dispatchable level [MW]\\
	$\vPowerTotal$ & Total power output [MW]\\
	$\vPowerTotalIn$ & Total charging power (storage unit) [MW]\\
	$\vPowerTotalOut$ & Total discharging power (storage unit) [MW]\\
	$\vReserveUp$ & Reserve provision [MW]\\
	$\vReserveUpViolation$ & Up-reserve constraint violation [MW]\\
	$\vEnergy$ & Energy output [MWh]\\
	$\vStorageUnitEnergy$ & Energy within storage unit [MWh]\\
	$\vLostLoadPower$ & Lost load power [MW]\\
	$\vLostLoadEnergy$ & Lost load energy [MWh]\\
	$\vPowerFlow$ & Powerflow over link connecting adjacent NEM zones [MW]\\
	& \\
	\multicolumn{2}{l}{\textbf{Parameters}}\\
	$\cOperatingCost$ & Total operating cost [\$]\\
	$\cOperatingCostThermal$ & Cost to operate thermal units [\$]\\
	$\cOperatingCostHydro$ & Cost to operate hydro units [\$]\\
	$\cOperatingCostWind$ & Cost to operate wind units [\$]\\
	$\cOperatingCostSolar$ & Cost to operate solar units  [\$]\\
	$\cOperatingCostStorage$ & Cost to operate storage units [\$]\\
	$\cOperatingCostReserveUpViolation$ & Cost of reserve violation [\$]\\
	$\cOperatingCostLostLoad$ & Lost load cost [\$]\\
	$\cLostLoadCost$ & Lost load penalty [\$/MWh]\\
	$\cReserveUpViolationPenalty$ & Reserve violation penalty [\$/MW]\\
	$\cMarginalCost$ & Short-run marginal cost [\$/MWh]\\
	$\cStartupCost$ & Normalised startup-cost [\$/MW]\\
	$\cShutdownCost$ & Normalised shutdown-cost [\$/MW]\\
	$\cEmissionsIntensity$ & Emissions intensity for generator $\iGenerator$ [tCO$_{2}$/MWh]\\
	$\cPowerOutputMax$ & Maximum power output [MW]\\
	$\cPowerOutputMin$ & Minimum power output [MW]\\
	$\cPowerOutputHydro$ & Hydro power output [MW]\\
	$\cPowerOutputWind$ & Wind power output [MW]\\
	$\cPowerOutputSolar$ & Solar power output [MW]\\
	$\cUpTimeMin$ & Minimum up time [hours]\\
	$\cDownTimeMin$ & Minimum down time [hours]\\
	$\cStartupDuration$ & Hours required to reach minimum dispatchable output following startup [h]\\
	$\cShutdownDuration$ & Hours required to transition from minimum dispatchable output to zero MW [h]\\
	$\cStartupTrajectory$ & Power output along startup trajectory [MW]\\
	$\cShutdownTrajectory$ & Power output along shutdown trajectory [MW]\\
	$\cRampRateUp$ & Ramp rate up (normal operation) [MW/h]\\
	$\cRampRateDown$ & Ramp down up (normal operation) [MW/h]\\
	$\cRampRateStartup$ & Ramp rate (startup) [MW/h]\\
	$\cRampRateShutdown$ & Ramp rate (shutdown) [MW/h]\\
	$\cPowerChargingMax$ & Storage unit maximum charging power [MW]\\
	$\cPowerDischargingMax$ & Storage unit maximum discharging power [MW]\\
	$\cStorageUnitEnergyMax$ & Storage unit maximum energy capacity [MWh]\\
	$\cStorageUnitEnergyIntervalEndMax$ & Storage unit maximum energy at end of operating scenario [MWh]\\
	$\cStorageUnitEnergyIntervalEndMin$ & Storage unit minimum energy at end of operating scenario [MWh]\\
	$\cStorageUnitEfficiencyCharging$ & Storage unit charging efficiency [--]\\
	$\cStorageUnitEfficiencyDischarging$ & Storage unit discharging efficiency [--]\\
	$\cDemand$ & Demand [MW]\\
	$\cReserveUpRequirement$ & Minimum up reserve requirement [MW]\\
	$\cIncidenceMatrix$ & Network incidence matrix [--]\\
	$\cPowerFlowMin$ & Minimum powerflow over link $\iLink$ [MW]\\
	$\cPowerFlowMax$ & Maximum powerflow over link $\iLink$ [MW]\\
	$\cScenarioProbibility$ & Probability of realising future energy path scenario $\iScenario$ for generator $\iGenerator$ [--]\\
	$\cCalibrationIntervalEnergy$ & Energy output for generator $\iGenerator$ under scenario $\iScenario$ in calibration interval $\iCalibrationInterval$ [MWh]\\
	$\cSchemeRevenue$ & Cumulative scheme revenue at beginning of calibration interval $\iCalibrationInterval$ [\$]\\
	$\cSchemeRevenueTarget$ & Target cumulative scheme revenue at end of calibration interval $\iCalibrationInterval$ [\$]\\
	$\cSchemeRevenueLowerBound$ & Lower bound for cumulative scheme revenue over forecast horizon [\$]\\
	$\cTotalCalibrationIntervals$ & Total calibration intervals [--]\\
	\hline
	\caption{Notation}
\end{longtable}

\section{Model}
A model predictive control (MPC) algorithm is developed to re-calibrate an emissions intensity baseline under a Refunded Emissions Payment (REP) scheme. Under the presented mechanism the baseline's value is communicated to generators prior to energy delivery. While this gives generators greater certainty with respect to their short-run marginal costs, the cost of doing so is means the scheme is no long revenue neutral (penalties could exceed credits or vice versa over the duration for which the baseline is fixed). The algorithm seeks to periodically update the baseline in order to manage scheme revenue, with these updates dependent upon current cumulative scheme revenue and expected energy output over future intervals for which the baseline will be fixed. Uncertainty is taken into account via stochastic programming, with the algorithm seeking to identify the path of baselines over future intervals that will reach a scheme revenue target while minimising deviations to the baseline's path. 

A unit commitment framework based on that presented in~\cite{morales-espana_tight_2015} is used to assess the suitability of this scheme, with a model for generic storage units following~\cite{pozo_unit_2014} also included. This model is used to simulate system operation over 2017-18 using a rolling-window approach similar to that presented in~\cite{riaz_computationally_2019-1}, illustrated in Figure~\ref{fig: model outline}. Given the computational complexity of the UC formulation a direct solution approach over the entire time horizon is computationally expensive, and often intractable. Instead, the model is solved in 41 hour blocks, with the last 17 hours of each block used as fixed initial conditions for the start of the following block. The MPC updating programming is applied periodically (weekly) to update REP scheme parameters. In this sense the presented can be thought of as using an agent-based framework that examines the interactions between the MPC updating protocol and the UC model solution, all while keeping track of emissions, scheme revenue, and energy output when progressing from one block to the next.

\begin{figure}
	\includegraphics{model_outline.pdf}
	\caption{Model outline - rolling-window UC approach to model longer time horizons. Policy parameters are periodically updated.}
	\label{fig: model outline}
\end{figure}

\subsection{Costs}

\subsubsection{UC objective}
The objective of the UC model is to meet system demand at the lowest cost for each block, denoted by $\cOperatingCost$:

\begin{equation}
	\minimise \quad \cOperatingCost
	\label{eqn: uc objective}
\end{equation}

With the total cost for each block comprised of expressions for the respective operating costs for each category of generators and constraint violation penalties within the system:

\begin{equation}
\cOperatingCost = \cOperatingCostThermal + \cOperatingCostHydro + \cOperatingCostWind + \cOperatingCostSolar + \cOperatingCostStorage + \cOperatingCostLostLoad + \cOperatingCostReserveUpViolation
\end{equation}

The constituent components for each expression are as follows. Thermal generators:
\begin{equation}
\cOperatingCostThermal = \sum\limits_{\iInterval} \sum\limits_{\iGenerator \in \sGeneratorsThermal} (\cMarginalCost + (\cEmissionsIntensity - \vBaseline)\vPermitPrice)\vEnergy + \left[\cStartupCost \cPowerOutputMax \vStartupIndicator + \cShutdownCost \cPowerOutputMax \vShutdownIndicator\right]
\label{eqn: total thermal generator operating cost}
\end{equation}

Hydro generators:
\begin{equation}
\cOperatingCostHydro = \sum\limits_{\iInterval}\sum\limits_{\iGenerator \in \sGeneratorsHydro}\cMarginalCost[\iGenerator] \vEnergy
\end{equation}

Wind generators:
\begin{equation}
\cOperatingCostWind = \sum\limits_{\iInterval}\sum\limits_{\iGenerator \in \sGeneratorsWind} \cMarginalCost[\iGenerator] \vEnergy
\end{equation}

Solar generators:
\begin{equation}
\cOperatingCostSolar = \sum\limits_{\iInterval}\sum\limits_{\iGenerator \in \sGeneratorsSolar} \cMarginalCost[\iGenerator] \vEnergy
\end{equation}

Storage units:
\begin{equation}
\cOperatingCostStorage = \sum\limits_{\iInterval}\sum\limits_{\iGenerator \in \sStorage} \cMarginalCost[\iGenerator] \vEnergy
\end{equation}

Value of lost load:
\begin{equation}
\cOperatingCostLostLoad = \sum\limits_{\iInterval}\sum\limits_{\iZone \in \sZones} \cLostLoadCost \vLostLoadEnergy
\end{equation}

Penalty for violating reserve requirements:
\begin{equation}
\cOperatingCostReserveUpViolation = \sum\limits_{\iInterval}\sum\limits_{\iRegion \in \sRegions} \cReserveUpViolationPenalty \vReserveUpViolation
\end{equation}

\subsection{Unit commitment constraints}
The objective given by~(\ref{eqn: uc objective}) is minimised with respect to the constraints that follow.

Ensures sufficient dispatchable power reserve in each region:
\begin{equation}
\sum\limits_{\iGenerator \in \sGeneratorsThermal_{\iRegion} \cup \sStorage_{\iRegion}} \vReserveUp + \vReserveUpViolation \geq \cReserveUpRequirement \quad  \iRegion \in \sRegions \quad  \iInterval \in \sIntervals
\label{eqn: reserve constraints}
\end{equation}

Coordinates generator state logic (on, off, start-up, shutdown):
\begin{equation}
\vOnIndicator - \vOnIndicator[\iGenerator,\iInterval-1] = \vStartupIndicator - \vShutdownIndicator \quad  \iGenerator \in \sGeneratorsThermal \quad  \iInterval \in \sIntervals
\end{equation}

Minimum up-time constraint:
\begin{equation}
\sum\limits_{\iIntervalAlias=\iInterval-\cUpTimeMin + 1}^{\iInterval} \vStartupIndicator[\iGenerator,\iIntervalAlias] \leq \vOnIndicator \quad  \iGenerator \in \sGeneratorsThermal  \quad  \iInterval \in \left[\cUpTimeMin, \iIntervalTerminal \right]
\end{equation}

Minimum down-time constraint:
\begin{equation}
\sum\limits_{\iIntervalAlias=\iInterval-\cDownTimeMin[\iGenerator]+1}^{\iInterval}\vShutdownIndicator[\iGenerator,\iIntervalAlias] \leq 1 - \vOnIndicator \quad  \iGenerator \in \sGeneratorsThermal  \quad  \iInterval \in \left[\cDownTimeMin, \iIntervalTerminal \right]
\end{equation}

Ramp-rate up constraint:
\begin{equation}
\left(\vPower + \vReserveUp\right) - \vPower[\iGenerator,\iInterval-1] \leq \cRampRateUp \quad  \iGenerator \in \sGeneratorsThermal  \quad  \iInterval \in \sIntervals
\end{equation}

Ramp-rate down constraint:
\begin{equation}
- \vPower + \vPower[\iGenerator,\iInterval-1] \leq \cRampRateDown \quad  \iGenerator \in \sGeneratorsThermal  \quad  \iInterval \in \sIntervals
\end{equation}

Ensure power production and reserves within minimum and maximum output limits for each generator:
\begin{align}
	\begin{split}
	\vPower + \vReserveUp \leq \left(\cPowerOutputMax[\iGenerator] - \cPowerOutputMin\right) \vOnIndicator - \left(\cPowerOutputMax[\iGenerator] -\cRampRateShutdown \right) \vShutdownIndicator[\iGenerator,\iInterval+1] & + \left(\cRampRateStartup - \cPowerOutputMin\right)\vStartupIndicator[\iGenerator,\iInterval+1]\\
	& \quad  \iGenerator \in \sGeneratorsThermal  \quad  \iInterval \in \sIntervals\\
	\end{split}
\end{align}

Total power output - quick-start units:
\begin{equation}
\vPowerTotal = \cPowerOutputMin \left(\vOnIndicator + \vStartupIndicator[\iGenerator,\iInterval+1]\right) + \vPower \quad  \iGenerator \in \sGeneratorsThermalQuickStart  \quad  \iInterval \in \sIntervals
\end{equation}

Total power output - slow-start units:
\begin{align}
\begin{split}
\vPowerTotal =  \cPowerOutputMin \left(\vOnIndicator + \vStartupIndicator[\iGenerator,\iInterval+1]\right) + & \vPower + \sum\limits_{\iIntervalAlias=1}^{\cStartupDuration} \cStartupTrajectory \vStartupIndicator[\iGenerator,\iInterval-\iIntervalAlias+\cStartupDuration+2] + \sum\limits_{\iIntervalAlias=2}^{\cShutdownDuration + 1} \cShutdownTrajectory \vShutdownIndicator[\iGenerator,\iInterval-\iIntervalAlias+2]\\
& \quad  \iGenerator \in \sGeneratorsThermalSlowStart  \quad  \iInterval \in \sIntervals
\end{split}
\end{align}

Note: There is an edge case where $\cRampRateStartup$ could be greater than $\cPowerOutputMax$, in which case $\vPowerTotal$ could be greater than the total installed capacity of the unit when it comes online. To prevent this from occurring the following constraints are added (these are not part of the original formulation given in~\cite{morales-espana_tight_2015}).

Ensure total power output + reserve requirement is less than installed capacity:
\begin{equation}
	0 \leq \vPowerTotal + \vReserveUp \leq \cPowerOutputMax[\iGenerator] \quad  \iGenerator \in \sGeneratorsThermal  \quad  \iInterval \in \sIntervals
\end{equation}

Total power output - wind units:
\begin{equation}
0 \leq \vPowerTotal \leq \cPowerOutputWind \quad  \iGenerator \in \sGeneratorsWind  \quad  \iInterval \in \sIntervals 
\end{equation}

Total power output - solar units:
\begin{equation}
0 \leq \vPowerTotal \leq \cPowerOutputSolar \quad  \iGenerator \in \sGeneratorsSolar  \quad  \iInterval \in \sIntervals 
\end{equation}

Total power output - hydro units:
\begin{equation}
0 \leq \vPowerTotal \leq \cPowerOutputHydro \quad  \iGenerator \in \sGeneratorsHydro  \quad  \iInterval \in \sIntervals 
\end{equation}

Constraints relating to generic storage units within a unit commitment framework follow those presented in~\cite{pozo_unit_2014}:
\begin{equation}
0 \leq \vPowerTotalIn \leq \cPowerChargingMax \quad  \iGenerator \in \sStorage  \quad  \iInterval \in \sIntervals
\end{equation}

\begin{equation}
0 \leq \vPowerTotalOut + \vReserveUp \leq \cPowerDischargingMax \quad  \iGenerator \in \sStorage  \quad  \iInterval \in \sIntervals
\end{equation}

\begin{equation}
0 \leq \vStorageUnitEnergy \leq \cStorageUnitEnergyMax \quad  \iGenerator \in \sStorage  \quad  \iInterval \in \sIntervals
\end{equation}

\begin{equation}
\vStorageUnitEnergy = \vStorageUnitEnergy[\iGenerator,\iInterval-1] + \cStorageUnitEfficiencyCharging \vPowerTotalIn - \frac{1}{\cStorageUnitEfficiencyDischarging} \vPowerTotalOut \quad  \iGenerator \in \sStorage  \quad  \iInterval \in \sIntervals
\end{equation}

\begin{equation}
\cStorageUnitEnergyIntervalEndMin \leq \vStorageUnitEnergy[\iGenerator,\iIntervalTerminal] \leq \cStorageUnitEnergyIntervalEndMax \quad  \iGenerator \in \sStorage  \quad  \iInterval \in \sIntervals 
\end{equation}

Power balance for each zone:
\begin{equation}
\sum\limits_{\iGenerator \in \sGenerators_{\iZone} \setminus \sStorage_{\iZone}} \vPowerTotal - \cDemand - \sum\limits_{\iLink \in \sLinks} \cIncidenceMatrix \vPowerFlow + \sum\limits_{\iGenerator \in \sStorage_{\iZone}} \left(\vPowerTotalOut - \vPowerTotalIn\right) + \vLostLoadPower = 0 \quad  \iZone \in \sZones  \quad \iInterval \in \sIntervals
\end{equation}

Non-negative lost-load:
\begin{equation}
\vLostLoadPower \geq 0 \quad  \iZone \in \sZones  \quad \iInterval \in \sIntervals
\end{equation}

Non-negative reserve:
\begin{equation}
\vReserveUp \geq 0 \quad  \iGenerator \in \iGenerator \in \sGeneratorsThermal_{\iRegion} \cup \sStorage_{\iRegion} \quad \iInterval \in \sIntervals
\end{equation}

Non-negative reserve violation:
\begin{equation}
\vReserveUpViolation \geq 0 \quad  \iZone \in \sZones  \quad \iInterval \in \sIntervals
\end{equation}

Power flow between zones:
\begin{equation}
\cPowerFlowMin \leq \vPowerFlow \leq \cPowerFlowMax \quad  \iLink \in \sLinks  \quad  \iInterval \in \sIntervals
\label{eqn: powerflow constraints}
\end{equation}

Energy from generator and storage units:
\begin{equation}
\vEnergy = \begin{dcases}
\frac{\vPowerTotal[\iGenerator, \iInterval-1] + \vPowerTotal}{2} &  \iGenerator \in \sGenerators \setminus \sStorage  \quad  \iInterval \in \sIntervals\\
\frac{\vPowerTotalOut[\iGenerator, \iInterval-1] + \vPowerTotalOut}{2} &  \iGenerator \in \sStorage  \quad  \iInterval \in \sIntervals\\
\end{dcases}
\end{equation}

Lost-load energy:
\begin{equation}
\vLostLoadEnergy = \frac{\vLostLoadPower[\iZone, \iInterval-1] + \vLostLoadPower}{2} \quad  \iZone \in \sZones  \quad  \iInterval \in \sIntervals
\label{eqn: lost-load energy constraint}
\end{equation}

\section{MPC model}
The MPC updating procedure is formulated as a mathematical program. The objective is to ensure that expected scheme revenue reaches some target, $\cSchemeRevenue[\iCalibrationInterval + \cTotalCalibrationIntervals]$, by the final calibration interval, $\iCalibrationInterval + \cTotalCalibrationIntervals$, while minimised changes to baseline's path over that interval. Constraint~(\ref{eqn: revenue target}) enforces this revenue target, while constraints~(\ref{eqn: lower revenue limit}) seek to prevent scheme revenue from dropping below a given limit at any point over the forecast horizon. Note that expected revenue, given by the summation terms in (\ref{eqn: revenue target}) and (\ref{eqn: lower revenue limit}) are with respect to generators that are under the scheme's remit. For most cases $\sGeneratorsEligible = \sGeneratorsThermal$ as only thermal generators will be covered by the scheme. This follows~\cite{aemc_integration_2016} which suggests the inclusion of legacy renewables under the scheme's remit would yield the owners of these assets windfall profits but would not result in material emissions abatement. New renewable plant would however be covered by the scheme, and included within $\sGeneratorsEligible$.

\begin{equation}
	\minimise\limits_{\vBaseline} \quad \sum\limits_{\iCalibrationInterval \in \sCalibrationIntervals} \left(\vBaseline[\iCalibrationInterval] - \vBaseline[\iCalibrationInterval-1]\right)^{2}
\end{equation}

subject to

\begin{equation}
	\cSchemeRevenue[\iCalibrationInterval] + \sum\limits_{\iCalibrationIntervalAlias=\iCalibrationInterval}^{\iCalibrationInterval + \cTotalCalibrationIntervals} \sum\limits_{\iGenerator \in \sGeneratorsEligible} \sum\limits_{\iScenario \in \sScenarios} \cScenarioProbibility \left(\cEmissionsIntensity - \vBaseline[\iCalibrationIntervalAlias]\right) \cCalibrationIntervalEnergy[\iGenerator,\iScenario,\iCalibrationIntervalAlias] \vPermitPrice[\iCalibrationIntervalAlias] = \cSchemeRevenue[\iCalibrationInterval+\cTotalCalibrationIntervals]
	\label{eqn: revenue target}
\end{equation}

\begin{equation}
	\cSchemeRevenue[\iCalibrationInterval] + \sum\limits_{\iCalibrationIntervalAlias=\iCalibrationInterval}^{\iCalibrationInterval + i} \sum\limits_{\iGenerator \in \sGeneratorsEligible} \sum\limits_{\iScenario \in \sScenarios} \cScenarioProbibility \left(\cEmissionsIntensity - \vBaseline[\iCalibrationIntervalAlias]\right) \cCalibrationIntervalEnergy[\iGenerator,\iScenario,\iCalibrationIntervalAlias] \vPermitPrice[\iCalibrationIntervalAlias] \geq \cSchemeRevenueLowerBound \quad   \forall i \in \left[1, 2, \ldots, \cTotalCalibrationIntervals \right]
	\label{eqn: lower revenue limit}
\end{equation}

\begin{equation}
	\vBaseline \geq 0 \quad  \iCalibrationInterval \in \sCalibrationIntervals
	\label{eqn: non-negative baseline}
\end{equation}

\bibliographystyle{elsarticle-num} 
\bibliography{derivation}

\end{document}
